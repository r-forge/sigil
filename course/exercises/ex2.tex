\documentclass[a4paper,12pt]{article}

\usepackage{vmargin}
\setpapersize[portrait]{A4}
%% \setmarginsrb{25mm}{10mm}{20mm}{20mm}% left, top, right, bottom
%% {12pt}{15mm}% head heigth / separation
%% {0pt}{15mm}% bottom height / separation
\setmargnohfrb{20mm}{20mm}{20mm}{20mm}

\setlength{\parindent}{0mm}
\setlength{\parskip}{\medskipamount}

\usepackage[english]{babel}
\usepackage[latin1]{inputenc}

\usepackage[T1]{fontenc}
\usepackage{textcomp}  % this can break some outline symbols in CM fonts, use only if absolutely necessary

%% \usepackage{stefan-fonts}  % commercial Charter fonts with full math support

% \usepackage{mathptmx}  % use Adobe Times as standard font with simulated math support
\usepackage[sc]{mathpazo}  % use Adobe Palatino as standard font with simulated math support
\usepackage{courier}

%% \usepackage{pifont}
%% \usepackage{eucal}

\usepackage{amsmath,amssymb,amsthm}
\usepackage{graphicx,rotating}
\usepackage{color}
\usepackage{array,hhline,booktabs}
\usepackage{xspace}
\usepackage{url}
\usepackage{alltt}
%% \usepackage{ifthen,calc,hyphenat}
%% \usepackage{pgf,pgfarrows,pgfnodes,pgfautomata,pgfheaps,pgfshade,xcolor,colortbl}

\newcommand{\REM}[1]{\textrm{\color[rgb]{.7,.2,.1}\# #1}}

\begin{document}

\emph{Marco Baroni \& Stefan Evert} \hfill %
{\small \url{http://purl.org/stefan.evert/SIGIL}}

\begin{center}
  \textbf{\Large Statistical Analysis of Corpus Data with R}

  \textbf{\large Exercise Sheet \#2}
\end{center}

In this exercise, you will familiarize yourself with the \emph{zipfR}
package, although, especially for the second part, some more general R
skills will also come in handy.  
% Please return a text file with the full list of commands you used, inserting
% answers to questions, observations etc.~as R comments (i.e., in lines that
% begin with \#).

\section{Type richness of German NP and PP constructions}

Although the notions of productivity and type richness have been used mostly
in morphology, similar ideas can also be applied to the study of syntax.  The
\emph{zipfR} package comes with data sets reporting frequency information
about the ``expansions'' of the German NP and PP phrases in the German TIGER
treebank (use \texttt{?TigerNP.spc} for more information).

Import the German NP and PP data sets. For the data set with more tokens,
compute a binomially interpolated vocabulary growth curve (VGC).  For the
smaller one, estimate an LNRE model (pick your favorite model, or try them
all) and use it to compute an expected VGC up to the size of the larger data
set.  Plot the interpolated and extrapolated VGCs, and determine which of the
two constructions appears to be more productive (in terms of type growth).


\section{Data lost with cut-off points}

Before you tackle this part of the exercise, read the second case study in
the \emph{zipfR} tutorial.

It is common, in collocation studies and similar work, to discard bigrams
(i.e., sequences of two words) below a certain occurrence threshold, typically
$f < 2$ (discarding bigrams that occur once, the \emph{hapax legomena}) and $f
< 5$ (discarding bigrams that occur 4 times or less).  In this exercise, we
try to assess the effect that such cuts have on the proportion of types
considered, on the basis of a sample from a relatively small corpus.

First, load the file \texttt{bigrams.100k.tfl} (containing bigrams extracted
from the first 100,000 tokens of the Brown corpus) as a \emph{zipfR} type
frequency list, and generate a frequency spectrum from this (see the
\emph{zipfR} documentation).

What proportion of bigram types occur only once? What proportion of types
occur 4 times or less?

Now, suppose that we want to use our data to estimate the proportion
of bigram types that would be lost by using the same two frequency
cut-offs if we had a 1 million word corpus.  How will you go about it?

\end{document}
