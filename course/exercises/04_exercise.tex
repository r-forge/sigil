\documentclass[a4paper,12pt]{article}

\usepackage{vmargin}
\setpapersize[portrait]{A4}
%% \setmarginsrb{25mm}{10mm}{20mm}{20mm}% left, top, right, bottom
%% {12pt}{15mm}% head heigth / separation
%% {0pt}{15mm}% bottom height / separation
\setmargnohfrb{20mm}{20mm}{20mm}{20mm}

\setlength{\parindent}{0mm}
\setlength{\parskip}{\medskipamount}

\usepackage[T1]{fontenc}
\usepackage[utf8]{inputenc}
\usepackage[english]{babel}

\usepackage[sc]{mathpazo}  % Adobe Palatino + Courier
\usepackage{courier}
\usepackage{textcomp} 

\usepackage{amsmath,amssymb,amsthm}
\usepackage{graphicx,rotating}
\usepackage{color}
\usepackage{array,hhline,booktabs}
\usepackage{xspace}
\usepackage{url}
\usepackage{alltt}
\usepackage{enumitem}

\usepackage{natbib}
\bibpunct{(}{)}{;}{a}{}{,}
\providecommand{\newblock}{}    % provide \newblock command if not in natbib.sty
\renewcommand{\bibsection}{}    % no separate section for references

%% same markup for R example code as in slides
\definecolor{osnared}{rgb}{.65,.05,.10}
\definecolor{osnablue}{rgb}{.03, .09, .56} 
\newcommand{\REM}[2][\small]{\textsf{#1\color{osnared}\# #2}}
\newenvironment{Rout}[1][\small]{%
  \begin{normalsize}#1\color{osnablue}\bfseries}{%
  \color{black}\mdseries\end{normalsize}}

%% set up PDF hyperlinks
\usepackage[colorlinks=true,linkcolor=osnared,urlcolor=osnablue,citecolor=black]{hyperref}


\begin{document}

\emph{Marco Baroni \& Stefan Evert} \hfill %
{\small \url{http://SIGIL.R-Forge.R-Project.org/}}

\begin{center}
  \textbf{\large Statistical Analysis of Corpus Data with R}

  \textbf{\large --- Exercise Sheet for Unit \#4 ---}
\end{center}

In the first part of this exercise, you will practise collocational analysis
as explained in the lecture slides by applying the procedure to a new data set
based on \emph{surface cooccurrence}.  The second part focuses on the
application of association measures to \emph{keyword extraction}, searching
for words that are particularly characteristic of spoken or written English.
The two data sets used for this exercise are part of the \texttt{corpora}
package.

\begin{itemize}
\item If you haven't done so already in Unit 1, download and install the
  \texttt{SIGIL} package from CRAN (see installation notes in handout from
  Unit 1).
\item We will use the \texttt{BNCInChargeOf} and \texttt{BNCcomparison} data
  sets included in this package. After loading the package with
  \texttt{library(SIGIL)}, familiarise yourself with the data sets by
  reading the respective help pages (with \texttt{?BNCInChargeOf} and
  \texttt{?BNCcomparison}).
\item The \texttt{BNCInChargeOf} data set contains surface collocates of
  the phrase \emph{in charge of}, extracted from the British National Corpus.
  Re-read the description of contingency tables
  for surface cooccurrences in the lecture slides, then calculate the
  contingency table of observed frequencies from the provided frequency
  information (\texttt{f.in}, \texttt{N.in}, \texttt{f.out}, \texttt{N.out}).
  Use \texttt{transform()} to add the new variables \texttt{O11},
  \texttt{O12}, \texttt{O21} and \texttt{O22} to the data set.
\item Compute the expected frequencies, row/column marginals, sample size, and
  association scores for a selection of measures, following the instructions
  in the lecture slides.  Rank the data set according to each association
  measure.  Which measure gives the intuitively most plausible ranking?
\item Association measures can also be used to identify characteristic
  \emph{keywords}, which are much more frequent in spoken than in written
  English, or vice versa.  The data set \texttt{BNCcomparison} lists the
  frequencies of a selection of English words in the written and spoken part
  of the British National Corpus.
\item Construct appropriate contingency tables for the frequency comparison
  setting, as explained in Unit 2.  First, determine the written and spoken
  sample sizes by summing over all rows of the data set.  Then calculate the
  observed frequencies \texttt{O11}, \texttt{O12}, \texttt{O21} and
  \texttt{O22} for each word (= row), and add them to the data set.
\item Which association measures might be sensible for keyword extraction?
  Compute the respective association scores using the same procedure as above,
  and rank the data set by \emph{keyness} for written or spoken English.  Do
  high/low association scores correspond to written or to spoken keyness?
  Compare the keywords identified by different measures.  Do you notice any
  specific problems of individual measures?
\end{itemize}

\end{document}
