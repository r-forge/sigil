%%%
%%% local configuration adjustments
%%%

%%% You can change pre-defined colours here, override built-in macros from the
%%% style definition and standard library, as well as define macros needed by
%%% all local documents.

%%% e.g. adjust counterpoint (dark green) for data projectors where greens are
%%% far too bright, as well as green component of light colour and pure green
%%% (of course, it's a better solution to adjust the gamma settings of your monitor)
%%
%% \definecolor{counterpoint}{rgb}{.1, .3, 0}
%% \definecolor{light}{rgb}{.45, .3, .55}
%% \definecolor{puregreen}{rgb}{0, .35, 0}

%% ----- extra packages we need to load

\usepackage{tikz}
\usepackage{alltt}              % code examples with nicely formatted comments
\usepackage{rotating}


%% ----- author and copyright messages (so updates are automatically inserted into all files)
\newcommand{\sigilauthors}{%
  \author[SIGIL]{Designed by Stefan Evert\inst{1} and Marco Baroni\inst{2}}
  \institute[Evert \& Baroni]{
    \inst{1}Computational Corpus Linguistics Group\\
    Friedrich-Alexander-Universit�t Erlangen-N�rnberg, Germany
    \and
    \inst{2}Center for Mind/Brain Sciences (CIMeC)\\
    University of Trento, Italy
  }
}
\newcommand{\sigilcopyright}{%
  \date[sigil.r-forge.r-project.org]{%
    \primary{\footnotesize\url{http://SIGIL.r-forge.r-project.org/}}\\
    \light{\tiny Copyright \textcopyright\ 2007--2018 Evert \& Baroni}}
}

%% ----- automatically show TOC reminder at beginning of each subsection
\AtBeginSubsection[]
{
  \begin{frame}
    \frametitle{Outline}
    \tableofcontents[current,currentsubsection]
  \end{frame}
}

%% ----- some useful macros for the SIGIL course

\newenvironment{Rcode}[1][]{%
  \setbeamercolor{block title}{fg=counterpoint,bg=counterpoint!15!white}%
  \setbeamercolor{block body}{bg=counterpoint!5!white}\small%
  \begin{block}{#1}\begin{alltt}\ungap[1]}{%
      \ungap[1]\end{alltt}\end{block}} % \end{alltt} ... to deconfuse emacs

%% use \sbox{\Rbox} ... \usebox{\Rbox} to insert arbitray latex into Rcode environment
\newsavebox{\Rbox}

%% > plot(x,y)      \REM{this produces a scatterplot}
\newcommand{\REM}[2][\small]{\textsf{#1\color{primary}\# #2}}

%% nice colour for R output: \begin{Rout} .. \end{Rout}
\newenvironment{Rout}[1][\footnotesize]{%
  \begin{footnotesize}#1\color{secondary}\bfseries}{%
    \color{black}\mdseries\end{footnotesize}}

%% rotated column labels for table (to fit long text into narrow columns
\newcommand{\rotLabel}[2][60]{\begin{rotate}{#1}#2\end{rotate}}
