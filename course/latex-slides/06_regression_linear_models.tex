\documentclass[t]{beamer}
%\documentclass[handout,t]{beamer}

\usetheme{StefanPlain}
\usepackage{beamer-tools-stefan} % standard packages, definitions and macros

%%%% uncomment the following macro libraries as needed
%% %%
%% support macros for graphics with the PGF packages
%%

%% \mmpaper[black]{0}{0}{8}{5}, \xypaper{0}{0}{8}{5}
%%   draw fine grid in pgf picture so that objects can easily be positioned
%%   \mmpaper draws cm/mm blocks, while \xypaper is based on (x,y) coordinates
\newcommand{\mmpaper}[5][black!40!white]{
  \begin{pgfscope}
    \color{#1}
    \pgfsetlinewidth{.2mm}
    \pgfgrid[step={\pgfpoint{1cm}{1cm}}]{\pgfpoint{#2cm}{#3cm}}{\pgfpoint{#4cm}{#5cm}}
    \pgfsetlinewidth{.1mm}
    \pgfgrid[step={\pgfpoint{1mm}{1mm}}]{\pgfpoint{#2cm}{#3cm}}{\pgfpoint{#4cm}{#5cm}}
  \end{pgfscope}
}
\newcommand{\xypaper}[5][black!40!white]{
  \begin{pgfscope}
    \color{#1}
    \pgfsetlinewidth{.2mm}
    \pgfgrid[step={\pgfxy(1,1)}]{\pgfxy(#2,#3)}{\pgfxy(#4,#5)}
    \pgfsetlinewidth{.1mm}
    \pgfgrid[step={\pgfxy(.1,.1)}]{\pgfxy(#2,#3)}{\pgfxy(#4,#5)}
  \end{pgfscope}
}

%%% Local Variables: 
%%% mode: latex
%%% TeX-master: ""
%%% End: 
   % basic PGF utility functions
%%
%% some macros for typesetting text
%%

%% \OPEN ... \CLOSE; \OPEN[np] ... \CLOSE[np]
%% bold large brackets for labelled bracketing notation
\newcommand<>{\OPEN}[1][]{\only#2{$\boldsymbol{\bigl[}\text{}_{\text{\raisebox{-2pt}{\textsc{#1}}}}$}}
\newcommand<>{\CLOSE}[1][]{\only#2{$\text{}_{\text{\raisebox{-2pt}{\textsc{#1}}}}\boldsymbol{\bigr]}$}}

%% \textgap ("_" representing missing letter)
\newcommand{\textgap}{\mbox{\hspace{.4pt}\texttt{\bfseries\secondary{\textunderscore}}\hspace{.4pt}}}

%% -- commands are defined in the hyperref bundle
%% \textstar, \textast (math \star and \ast symbols in text mode, with some extra spacing)
% \newcommand{\textstar}{$\mspace{.8mu}\star\mspace{.8mu}$}
% \newcommand{\textast}{$\ast$}

%% $\p{\ctext{abc}}$ (cited text in mathematical equations, e.g. n-gram probabilities)
\newcommand{\ctext}[1]{\text{\textcite{#1}}}

%% $\p{\btext{abc}}$ (normal black text even in coloured math environment)
\newcommand{\btext}[1]{\text{\foreground{#1}}} 

%% text subscripts and superscripts (can be used in math and text mode)
\newcommand{\tsup}[1]{\ensuremath{^{\text{#1}}}}
\newcommand{\tsub}[1]{\ensuremath{_{\text{#1}}}}

%%% Local Variables: 
%%% mode: latex
%%% TeX-master: ""
%%% End: 
  % some macros for typesetting text
%%
%%  INCLUDE: math.tex
%%  
%%  basic mathematical symbols and constructs (not specific to cooccurrences)
%%


%% \setN, \setN[0], \setZ, \setQ, \setR, \setC
%% abbreviations for common number spaces
\newcommand{\setN}[1][]{\mathbb{N}_{#1}} % allows \setN and \setN[0]
\newcommand{\setZ}{\mathbb{Z}}
\newcommand{\setQ}{\mathbb{Q}}
\newcommand{\setR}{\mathbb{R}}
\newcommand{\setC}{\mathbb{C}}

%% \set{el_1, el_2, ...};  \setdef{el}{condition};  \bigset{..}, \bigsetdef{..}{..}
%% extensional and intensional definition of sets, with "big" versions (like \bigl etc.)
\newcommand{\set}[1]{\left\{#1\right\}}
\newcommand{\setdef}[2]{\set{#1\,\left|\,#2\right.}}
\newcommand{\bigset}[1]{\bigl\{#1\bigr\}}
\newcommand{\bigsetdef}[2]{\bigset{#1\bigm|#2}}
\newcommand{\Bigset}[1]{\Bigl\{#1\Bigr\}}
\newcommand{\Bigsetdef}[2]{\Bigset{#1\Bigm|#2}}
\newcommand{\biggset}[1]{\biggl\{#1\biggr\}}
\newcommand{\biggsetdef}[2]{\biggset{#1\biggm|#2}}

%% \compl{X} = complement of set X
\newcommand{\compl}[1]{\mathcal{C} #1}

%% \eps == \epsilon, \si == \sigma, \sisi == \sigma^2, \ka == \kappa
\newcommand{\eps}{\epsilon}
\newcommand{\si}{\sigma}
\newcommand{\sisi}{\sigma^2}
\newcommand{\ka}{\kappa}

%% \abs{expr}, \bigabs{expr}, \norm{expr}, \bignorm{expr}
%% absolute value and norm of expression, with "big" versions
\newcommand{\abs}[1]{\left\lvert#1\right\rvert}
\newcommand{\bigabs}[1]{\bigl\lvert#1\bigr\rvert}
\newcommand{\norm}[1]{\left\lVert#1\right\rVert}
\newcommand{\bignorm}[1]{\bigl\lVert#1\bigr\rVert}

%% \constpi == constant PI (in bold font)
\newcommand{\constpi}{\boldsymbol{\pi}}

%% \dx == "dx";  \dx[z] == "dz";  \dpi == \dx[\pi];  
%% \dG == \dx[G], \dt == \dx[t]
\newcommand{\dx}[1][x]{\,d#1}
\newcommand{\dpi}{\dx[\pi]}
\newcommand{\dG}{\dx[G]}
\newcommand{\dt}{\dx[t]}

%% \Int{\frac{1}{2} x^2}_a^b
%% anti-derivative evaluated to compute definite integral
\newcommand{\Int}[1]{\left[#1\right]}

%% \limdownto{x}{0}
%% limit from above for x -> 0
\newcommand{\limdownto}[2]{\lim_{#1\,\downarrow\,#2}}

%% \iffdef == ":<=>";  \iffdefR == "<=>:"
\newcommand{\iffdef}{\;:\!\iff}
\newcommand{\iffdefR}{\iff\!:\;}

%% \logten(x) 
%% base 10 logarithm, which is always used in the UCS system
\newcommand{\logten}{\log_{10}}

%% \e+3, \e-6, \e-{12}, 5.5\x\e-3
%% engineering-style notation (orders of magnitude) for floating-point numbers
\newcommand{\e}[2]{10^{\ifthenelse{\equal{#1}{+}}{}{#1}#2}}
\newcommand{\x}{\cdot}

%% \Landau{ n^2 }, \bigLandau{ N^2 }
%% Landau symbol ("big oh notation")
\newcommand{\Landau}[1]{\mathcal{O}\left({#1}\right)}
\newcommand{\bigLandau}[1]{\mathcal{O}\bigl({#1}\bigr)}


%%% Local Variables: 
%%% mode: latex
%%% TeX-master: t
%%% End: 
  % basic mathematical notation
%%
%%  INCLUDE: stat.tex
%%  
%%  symbols and notation for probability theory and statistics
%%


%% \p{X=k};  \pC{X=k}{Y=l};  \pscale{\frac{Z}{S^2}}
%% probability P(X=k), conditional probability P(X=k|Y=l), and variants with scaled parentheses
\newcommand{\p}[1]{\mathop{\mathrm{Pr}}\bigl(#1\bigr)}
\newcommand{\pscale}[1]{\mathop{\mathrm{Pr}}\left(#1\right)}
\newcommand{\pC}[2]{\p{#1\bigm|#2}} 
\newcommand{\pCscale}[2]{\pscale{#1\,\left|\,#2\right.}} 

%% \Exp{X};  \Var{X};  \Exp[0]{X};  \Var[0]{X};  \Expscale{X};  \Varscale{X}
%% expectation E[X] and variance V[X], expectation and variance under null hypothesis, 
%% and variants with scaled brackets
\newcommand{\Exp}[2][]{E_{#1}\bigl[#2\bigr]}
\newcommand{\Var}[2][]{\mathop{\mathrm{Var}}_{#1}\bigl[#2\bigr]}
\newcommand{\Expscale}[2][]{E_{#1}\left[#2\right]}
\newcommand{\Varscale}[2][]{\mathop{\mathrm{Var}}_{#1}\left[#2\right]}

%% \I{f_i = M};  \bigI{\sum_{i=1}^S f_i = N}
%% indicator variable (as in Baayen 2001), and variant with explicitly scaled brackets
\newcommand{\I}[1]{I_{\left[#1\right]}}
\newcommand{\bigI}[1]{I_{\bigl[#1\bigr]}}

%% \Hind;  \Hhom;  \Hnull{\kappa = x}
%% null hypothesis of independence and homogeneity; general null hypothesis identified by condition
\newcommand{\Hind}{H_0}
\newcommand{\Hhom}{H_{0,\, hom}}
\newcommand{\Hnull}[1]{H_{#1}}

%% \confint{\kappa};  \confint[0.99]{\kappa}
%% confidence interval for specified population characteristic (conf. level defaults to \alpha)
\newcommand{\confint}[2][\alpha]{I_{#2,\,#1}}

%% \df = 1
%% degrees of freedom
\newcommand{\df}{\mathit{df}}


%%% Local Variables: 
%%% mode: latex
%%% TeX-master: t
%%% End: 
  % notation for probability theory and statistics
%%
%% convenience macros for linear algebra (vectors and matrices)
%%

%% \Vector[i]{x} ... vector variable with optional _superscript_ index in parentheses
%% \Vector[']{x} ... special case: ' superscript not enclosed in parentheses
%% \vx, \vy, \vz ... abbreviations for common vector names
\newcommand{\Vector}[2][]{\mathbf{#2}\ifthenelse{\equal{#1}{}}{}{^{(#1)}}}
\newcommand{\vx}[1][]{\Vector[#1]{x}}
\newcommand{\vy}[1][]{\Vector[#1]{y}}
\newcommand{\vz}[1][]{\Vector[#1]{z}}
\newcommand{\vu}[1][]{\Vector[#1]{u}}
\newcommand{\vv}[1][]{\Vector[#1]{v}}
\newcommand{\vw}[1][]{\Vector[#1]{w}}
\newcommand{\vm}[1][]{\Vector[#1]{m}} 
\newcommand{\va}[1][]{\Vector[#1]{a}} % vectors of coefficients
\newcommand{\vb}[1][]{\Vector[#1]{b}} % for basis
\newcommand{\ve}[1][]{\Vector[#1]{e}} % for standard basis of R^n
\newcommand{\vn}[1][]{\Vector[#1]{n}} % normal vector
\newcommand{\vnull}[1][]{\Vector[#1]{0}} % neutral element

%% \Span{\vb[1],\ldots,\vb[k]} ... span of set of vectors
%% \Rank{...} ... rank of set of vectors or matrix
%% \Det{...}, \det A ... determinant of a set of vectors / a matrix A
%% \Image{f}, \Kernel{f} ... image and kernel of a linear map
\newcommand{\Span}[1]{\mathop{\text{sp}}\left(#1\right)}
\newcommand{\Rank}[1]{\mathop{\text{rank}}\left(#1\right)}
\newcommand{\Det}[1]{\mathop{\text{Det}}\left(#1\right)}
%% \det is already defined in the standard library
\newcommand{\Image}[1]{\mathop{\text{Im}}\left(#1\right)}
\newcommand{\Kernel}[1]{\mathop{\text{Ker}}\left(#1\right)}

%% \dist[2]{\vx}{\vy} ... distance between two vectors (p-metric)
\newcommand{\dist}[3][]{d_{#1}\left(#2, #3\right)}
\newcommand{\bigdist}[3][]{d_{#1}\bigl(#2, #3\bigr)}

%% \sprod{\vu}{\vv} ... scalar product
\newcommand{\sprod}[2]{\left\langle #1, #2 \right\rangle}
\newcommand{\bigsprod}[2]{\bigl\langle #1, #2 \bigr\rangle}


%%% Local Variables: 
%%% mode: latex
%%% TeX-master: ""
%%% End: 
% convenience macros for vectors and matrices
%% %%
%% grid-like layout of strings etc in PGF picture
%%

\usepackage{calc}

%% configure grid layout and style:
%%   \setGridOrigin{<x>}{<y>} ... origin of grid
%%   \setGridX{3cm}           ... unit on x-axis (columns)
%%   \setGridY{1cm}           ... unit on y-axis (rows)
%%   \setGridAlignment{left}{base} ... alignment of elements in grid
%%   \renewcommand{\gridStyle}[1]{\small\secondary{#1}}
\newlength{\gridOriginX}  \setlength{\gridOriginX}{0cm}
\newlength{\gridOriginY}  \setlength{\gridOriginY}{0cm}
\newlength{\gridXunit}    \setlength{\gridXunit}{3cm}
\newlength{\gridYunit}    \setlength{\gridYunit}{1em}
\newcommand{\setGridOrigin}[2]{%
  \setlength{\gridOriginX}{#1}%
  \setlength{\gridOriginY}{#2}%
}
\newcommand{\setGridX}[1]{\setlength{\gridXunit}{#1}}
\newcommand{\setGridY}[1]{\setlength{\gridYunit}{#1}}
\newcommand{\gridAlignH}{left}
\newcommand{\gridAlignV}{base}
\newcommand{\setGridAlignment}[2]{%
  \renewcommand{\gridAlignH}{#1}%
  \renewcommand{\gridAlignV}{#2}%
}
\newcommand{\gridStyle}[1]{\secondary{\small{}#1}}

%% these are temporary registers for internal use
\newlength{\gridTemp}
\newlength{\gridTempX}
\newlength{\gridTempY}

%% internal: \gridXYcell{<col>}{<row>} -> \gridTempX, \gridTempY
\newcommand{\gridXYcell}[2]{%
  \setlength{\gridTempX}{\gridXunit * \real{#1} + \gridOriginX}%
  \setlength{\gridTempY}{\gridYunit * \real{#2} + \gridOriginY}%
}

%% print text (\pgfbox) in specified cell (Col, Row)
%%   \gridText<overlay>{Col}{Row}{Text}
\newcommand<>{\gridText}[3]{%
  \only#4{%
    \gridXYcell{#1}{#2}%
    \pgfputat{\pgfpoint{\gridTempX}{\gridTempY}}{%
      \pgfbox[\gridAlignH,\gridAlignV]{\gridStyle{#3}}}}%
}


%%% Local Variables: 
%%% mode: latex
%%% TeX-master: ""
%%% End: 
  % grid-like layout of text and graphics in PGF picture
%% %%
%% macros for typesetting parse trees with PGF
%%

\usepackage{calc}

%% configure tree dimensions:
%%   \setlength{\treeHstep}{5mm}, \setlength{\treeVstep}{5mm}
%%   \setTreeHeight{6} -> 6 non-terminal layers above terminal string (default: 8)
%%   \setTreeRelative{3.5} -> slots are specified relative to this position (default: 0)
%%   \setTreeBoxW{Text}, \setTreeBoxH{Text}, \setTreeBoxWH{Text} 
%%      -> set amount of space reserved for tree node (to accommodate Text in current font)
%%
\newlength{\treeHstep}
\setlength{\treeHstep}{10mm}
\newlength{\treeVstep}
\setlength{\treeVstep}{10mm}
\newlength{\treeBoxH}
\newlength{\treeBoxD}
\newlength{\treeBoxW}
%% internal lengths and commands
\newlength{\treeBoxTotalH}
\newlength{\treeBoxBaseDY}
\newcommand{\calculateTreeBox}{%
  \setlength{\treeBoxTotalH}{\treeBoxH + \treeBoxD}%
  \setlength{\treeBoxBaseDY}{\treeBoxD - \treeBoxTotalH / 2}%
}
\newcommand{\setTreeBoxW}[1]{\settowidth{\treeBoxW}{#1}}
\newcommand{\setTreeBoxH}[1]{%
  \settoheight{\treeBoxH}{#1}%
  \settodepth{\treeBoxD}{#1}%
  \calculateTreeBox{}}
\newcommand{\setTreeBoxWH}[1]{\setTreeBoxW{#1}\setTreeBoxH{#1}}
\setTreeBoxW{XX}                % default: W = two wide letters
\setTreeBoxH{Tg}                % default: H = full height + depth
\newcounter{TreeHeight}
\setcounter{TreeHeight}{8}
\newcommand{\setTreeHeight}[1]{\setcounter{TreeHeight}{#1}}
%% internal lengths and commands
\newlength{\treeRelativeX}
\setlength{\treeRelativeX}{0mm}
\newcommand{\setTreeRelative}[1]{\setlength{\treeRelativeX}{\treeHstep * \real{#1}}}

%% these are temporary registers for internal use
\newlength{\treeTemp}
\newlength{\treeTempX}
\newlength{\treeTempY}
\newcounter{TreeTemp}

%% internal: \treeYforLayer{<layer>} -> \treeTempY = y-position
\newcommand{\treeYforLayer}[1]{%
  \setlength{\treeTempY}{\treeVstep * \real{\theTreeHeight} - \treeVstep * \real{#1}}%
}

%% internal: \treeXforSlot{<slot>} -> \treeTempX = x-position
\newcommand{\treeXforSlot}[1]{%
  \setlength{\treeTempX}{\treeRelativeX + \treeHstep * \real{#1}}%
}

%% internal: insert strut for full text line height (to improve text alignment)
\newcommand{\treeStrut}{%
  \settoheight{\treeTemp}{X}\rule{0mm}{\treeTemp}%
}

%% put (non-terminal) tree node in slot <Slot> on layer <Layer>, 
%% labelled with <Text> and assigned the node label <Lab>; 
%% layers count from 0 = root layer at the top of the tree
%%   \treeNode<overlay>{Lab}{Layer}{Slot}{Text}
%%   \treeNodePT<overlay>{Lab}{Slot}{Text} ... on pre-terminal layer
\newcommand<>{\treeNode}[4]{%
  \treeYforLayer{#2}\treeXforSlot{#3}%
  \pgfnoderect{#1}[virtual]{\pgfpoint{\treeTempX}{\treeTempY}}{\pgfpoint{\treeBoxW}{\treeBoxTotalH}}%
  \addtolength{\treeTempY}{\treeBoxBaseDY}%
  \only#5{\pgfputat{\pgfpoint{\treeTempX}{\treeTempY}}{\pgfbox[center,base]{#4}}}%
}
\newcommand<>{\treeNodePT}[3]{%
  \setcounter{TreeTemp}{\theTreeHeight - 1}%
  \treeNode#4{#1}{\theTreeTemp}{#2}{#3}%
}

%% put (terminal) leaf node in slot <Slot> on lexical layer>, 
%% labelled with <Text> and assigned the node label <Lab>
%%   \treeLeaf<overlay>{Lab}{Slot}{Text}
\newcommand<>{\treeLeaf}[3]{%
  \treeNode#4{#1}{\theTreeHeight}{#2}{#3}%
}

%% connect tree nodes with straight (unlabelled) edge
%%   \treeEdge{Lab1}{Lab2}
\newcommand<>{\treeEdge}[2]{%
  \begin{pgfscope}
    \pgfclearstartarrow
    \pgfclearendarrow
    \pgfsetlinewidth{1pt}
    \pgfnodeconnline#3{#1}{#2}
  \end{pgfscope}
}

%% connect tree nodes with curved (unlabelled) edge; leaves upper node
%% Lab1 at Angle relative to vertical, enters lower node Lab2 vertically
%%   \treeEdgeCurve{Lab1}{Angle}{Lab2}
\newcommand<>{\treeEdgeCurve}[3]{%
  \begin{pgfscope}
    \pgfclearstartarrow%
    \pgfclearendarrow%
    \pgfsetlinewidth{1pt}%
    \setcounter{TreeTemp}{#2 - 90}%
    \pgfnodeconncurve#4{#1}{#3}{\theTreeTemp}{90}{\treeVstep}{\treeVstep}
  \end{pgfscope}
}

%% attach note to given node (in right bottom corner)
%%   \treeNote<overlay>[right]{Lab}{Text}
\newcommand<>{\treeNote}[3][left]{%
  \only#4{%
    \pgfputat{\pgfnodeborder{#2}{-60}{-2pt}}{%
      \pgfbox[#1,top]{\setlength{\fboxsep}{1pt}\colorbox{white}{%
          \scriptsize\secondary{\treeStrut{}#3}}}}}%
}

%% represent subtree (root=Root, boundary=Left..Right) as triangle
%%   \treeTriangle<overlay>{Root}{Left}{Right}
\newcommand<>{\treeTriangle}[3]{%
  \begin{pgfscope}
    \only#4{%
      \pgfclearstartarrow%
      \pgfclearendarrow%
      \pgfsetlinewidth{1pt}%
      \pgfmoveto{\pgfnodeborder{#1}{-90}{2pt}}%
      \pgflineto{\pgfnodeborder{#2}{135}{2pt}}%
      \pgflineto{\pgfnodeborder{#3}{45}{2pt}}%
      \pgfclosestroke%
    }
  \end{pgfscope}
}



%%% Local Variables: 
%%% mode: latex
%%% TeX-master: ""
%%% End: 
  % typesetting parse trees with PGF
%% %%
%% macros for demonstrating chart parsers with PGF
%%

\usepackage{calc,ifthen}

%% configure chart dimensions:
%%   \setlength{\chartXunit}{5mm}, \setlength{\chartYunit}{5mm}
%%   \setChartWidth{8}
\newlength{\chartXunit}
\setlength{\chartXunit}{10mm}
\newlength{\chartYunit}
\setlength{\chartYunit}{5mm}
\newcounter{ChartWidth}
\setcounter{ChartWidth}{8}
\newcommand{\setChartWidth}[1]{%
  \setcounter{ChartWidth}{#1}}

%% these are temporary registers for internal use
\newlength{\chartTemp}
\newlength{\chartTempH}
\newlength{\chartTempW}
\newlength{\chartTempX}
\newlength{\chartTempY}

%% internal: \chartXcoord{<x>} -> \chartTempX
\newcommand{\chartXcoord}[1]{%
  \setlength{\chartTempX}{\chartXunit * \real{#1}}%
}

%% internal: \chartYcoord{<y>} -> \chartTempY
\newcommand{\chartYcoord}[1]{%
  \setlength{\chartTempY}{\chartYunit * \real{#1}}%
}

%% internal: \chartXYcoord{<x>}{<y>} -> \chartTempX, \chartTempY
\newcommand{\chartXYcoord}[2]{%
  \chartXcoord{#1}\chartYcoord{#2}%
}

%% internal: insert strut for full text line height (to improve text alignment)
\newcommand{\chartStrut}{%
  \settoheight{\chartTemp}{X}\rule{0mm}{\chartTemp}%
}
 
%% put terminal symbol <Text> in given <Slot> (x-coordinate) with optional label
%%   \chartTerminal<overlay>[Lab]{Slot}{Text}
\newcommand<>{\chartTerminal}[3][chartNULL]{%
  \only#4{%
    \chartXYcoord{#2}{0}
    \settowidth{\chartTempW}{#3}
    \pgfputat{\pgfpoint{\chartTempX}{\chartTempY}}{\pgfbox[center,base]{\chartStrut{}#3}}%
    \addtolength{\chartTempY}{.4em}%
    \pgfnoderect{#1}[virtual]{\pgfpoint{\chartTempX}{\chartTempY}}{%
      \pgfpoint{\chartXunit / 2}{1em}}%
  }
}

%% draw arrow across span of substring at height <Y> and label with <Tag>
%%   \chartSpan[Lab]{Y}{X1}{X2}{Tag}
\newcommand<>{\chartSpan}[5][chartNULL]{%
  \only#6{%
    \chartXYcoord{#3}{#2}%
    \setlength{\chartTemp}{\chartTempX}%
    \chartXcoord{#4}%
    \addtolength{\chartTemp}{\chartXunit * \real{-.35}}%
    \addtolength{\chartTempX}{\chartXunit * \real{.35}}%
    \begin{pgfscope}
      \pgfclearstartarrow{}\pgfsetendarrow{\pgfarrowtriangle{3pt}}%
      \pgfsetlinewidth{1pt}%
      \color{counterpoint}%
      \pgfline{\pgfpoint{\chartTemp}{\chartTempY}}{\pgfpoint{\chartTempX}{\chartTempY}}%
    \end{pgfscope}
    \setlength{\chartTempH}{\chartTempY + \chartYunit * \real{0.4}}%
    \setlength{\chartTempW}{\chartTempX - \chartTemp}%
    \setlength{\chartTempX}{(\chartTemp + \chartTempX) / 2}%
    \pgfnoderect{#1}[virtual]{\pgfpoint{\chartTempX}{\chartTempH}}{\pgfpoint{\chartTempW}{\chartYunit * \real{0.8}}}%
    \addtolength{\chartTempY}{2pt}%
    \pgfputat{\pgfpoint{\chartTempX}{\chartTempY}}{%
      \pgfbox[center,base]{\scriptsize\color{secondary}#5}}%
  }%  
}

%% CFG production for use as span label (LHS in bold)
%%   \charProd{LHS}{RHS}
\newcommand{\chartProd}[2]{%
  \textbf{#1} $\to$ {#2}%
}

%% draw dashed line across entire chart at height <Y>, optional with <Text> on LHS
%%   \chartRule{Y}, \chartRule[Text]{Y}
\newcommand<>{\chartRule}[2][]{%
  \only#3{%
    \begin{pgfscope}
      \chartXYcoord{0}{#2}%
      \setlength{\chartTemp}{\chartTempX}%
      \chartXcoord{\theChartWidth}%
      \addtolength{\chartTemp}{\chartXunit * \real{-.5}}%
      \addtolength{\chartTempX}{\chartXunit * \real{-.5}}%
      \addtolength{\chartTempY}{\chartYunit * \real{.4}}%
      \pgfclearstartarrow{}\pgfclearendarrow{}%
      \pgfsetlinewidth{1pt}%
      \pgfsetdash{{2mm}{2mm}}{1mm}%
      \pgfline{\pgfpoint{\chartTemp}{\chartTempY}}{\pgfpoint{\chartTempX}{\chartTempY}}%
      \addtolength{\chartTemp}{-4pt}%
      \pgfputat{\pgfpoint{\chartTemp}{\chartTempY}}{\pgfbox[right,base]{\small #1}}%
    \end{pgfscope}
  }%  
}

%% draw edge of parse tree into chart (nodes must have been labelled);
%% uses current color (change with \color{...} before drawing edges)
%%   \chartEdge<overlay>{Lab1}{Lab2}
\newcommand<>{\chartEdge}[2]{%
  \begin{pgfscope}
    \pgfclearstartarrow%
    \pgfclearendarrow%
    \pgfsetlinewidth{1.5pt}%
    \pgfnodesetsepstart{2pt}%
    \pgfnodeconnline#3{#1}{#2}%
  \end{pgfscope}
}

%%% Local Variables: 
%%% mode: latex
%%% TeX-master: ""
%%% End: 
 % demonstrating chart parsers with PGF
%% %%
%% some useful macros for formal languages
%%

%% empty word \eps (length \abs{w} defined in math.tex)
\newcommand{\eps}{\epsilon}

%% roman (or sans-serif) letters used as symbols in examples
\newcommand{\z}[1]{\textsf{#1}}

%% language denoted by a given grammar or regular expression
\newcommand{\lang}[2][]{\mathcal{L}_{#1}\bigl[#2\bigr]}

%% language generated by a Turing machine 
\newcommand{\generate}[2][]{\mathcal{G}_{#1}\bigl[#2\bigr]}

%% regular expressions: \reg{expr}, \R(, \R), \R*, ..., \len{\reg{expr}}
\newcommand{\R}[1]{\texttt{#1}}
\newcommand{\reg}[1]{\underline{#1}}
\newcommand{\regeps}{\reg{\eps}}
%\newcommand{\len}[1]{\ell\left(#1\right)} %% do we still need that command?

%% class \langReg[\Sigma] of all regular languages, and set \langR[\Sigma] of all regular expressions
\newcommand{\langReg}[1][\Sigma]{\operatorname{Reg}(#1)}
\newcommand{\langR}[1][\Sigma]{R(#1)}


%%
%% Endliche Automaten
%%


%% Regeln f�r die formale Beschreibung (im Formelmodus zu verwenden):
%%   \Xearule{<lab1>}{<lab2>}{<label>}
%% [<lab1>, <lab2> sind Zustandslabel und werden im Formelmodus gesetzt]
%% [<label> bezeichnet das Eingabezeichen und wird AUCH im FORMELMODUS gesetzt]
%%   \earule{<n1>}{<n2>}{<label>}
%% [<label> bezeichnet das Label einer Bewegung und wird im Textmodus gesetzt]
\newcommand{\earule}[3]{q_{#1} \overset{\text{#3}}{\longrightarrow} q_{#2}}
\newcommand{\Xearule}[3]{#1 \overset{#3}{\longrightarrow} #2}



%%
%% Kontextfreie Grammatiken
%%

%% Ableitungsschritte und Ableitung
%%   \drv[<grammar>]
%%   \Drv[<grammar>]
\newcommand{\drv}[1][]{\implies_{#1}}
\newcommand{\Drv}[1][]{\implies_{#1}^*}

%% class \langKF[\Sigma] of all context-free languages
\newcommand{\langKF}[1][\Sigma]{\operatorname{KF}(#1)}

%% Regeln f"ur die formale Beschreibung von Kellerautomaten (im Formelmodus):
%%   \Xpdarule{<state1>}{<V>}{<label>}{<state2>}{<stack>}
%% [�bergang von Zustand <state1> mit oberem Kellersymbol <V> in
%%  Zustand <state2>, wobei <stack> auf dem Keller abgelegt wird;
%%  dabei wird das Zeichen <label> verarbeitet]
\newcommand{\Xpdarule}[5]{(#1,#2) \overset{#3}{\longrightarrow} (#4,#5)}

%% A \to \alpha \oder \beta ... Disjunktionsstrich mit zus�tzlichem Abstand
\newcommand{\oder}{\,|\,}



%%% Local Variables: 
%%% mode: latex
%%% TeX-master: ""
%%% End: 
    % some notation for formal language theory

\renewcommand{\bar}[1]{\overline{#1}}

%%%%%%%%%%%%%%%%%%%%%%%%%%%%%%%%%%%%%%%%%%%%%%%%%%%%%%%%%%%%%%%%%%%%%%
%% Titlepage

\title[SIGIL: Linear Models]{Statistical Analysis of Corpus Data with R}
\subtitle{A short introduction to regression and linear models}

\author[Baroni \& Evert]{Designed by Marco Baroni\inst{1} and Stefan Evert\inst{2}}
\institute[Trento/Osnabr\"uck]{
  \inst{1}Center for Mind/Brain Sciences (CIMeC)\\
  University of Trento
  \and
  \inst{2}Institute of Cognitive Science (IKW)\\
  University of Onsabr�ck
}
\date{}


\begin{document}

\frame{\titlepage}

%%%%%%%%%%%%%%%%%%%%%%%%%%%%%%%%%%%%%%%%%%%%%%%%%%%%%%%%%%%%%%%%%%%%%%
%% Outline

\section*{Outline}
\frame{ 
  \frametitle{Outline}
  \tableofcontents
}

%%%%%%%%%%%%%%%%%%%%%%%%%%%%%%%%%%%%%%%%%%%%%%%%%%%%%%%%%%%%%%%%%%%%%%
\section{Regression}

\subsection{Simple linear regression}

\begin{frame}
  \frametitle{Linear regression}
  %\framesubtitle{}

  \begin{itemize}
  \item Can random variable $Y$ be predicted from r.~v.\ $X$?
    \begin{itemize}
    \item[\hand] focus on linear relationship between variables
    \item[]
    \end{itemize}
  \item Linear predictor: 
    \[
    Y \approx \beta_0 + \beta_1\cdot X
    \]
    \ungap[1]
    \begin{itemize}
    \item $\beta_0$ = intercept of regression line
    \item $\beta_1$ = slope of regression line
    \item[]
    \end{itemize}
    \pause
  \item Least-squares regression minimizes prediction error
    \[
    Q = \sum_{i=1}^n \bigl[ y_i - (\beta_0 + \beta_1 x_i) \bigr]^2
    \]
    for data points $(x_1,y_1), \ldots, (x_n, y_n)$
  \end{itemize}
\end{frame}

\begin{frame}
  \frametitle{Simple linear regression}
  %\framesubtitle{}

  \begin{itemize}
  \item Coefficients of least-squares line
    \begin{align*}
      \hat{\beta}_1 &= \frac{
        \sum_{i=1}^n x_i y_i - n \bar{x}_n \bar{y}_n 
      }{
        \sum_{i=1}^n x_i^2 - n \bar{x}_n^2
      }
      \\[2mm]
      \hat{\beta}_0 &= \bar{y}_n - \hat{\beta}_1 \bar{x}_n
    \end{align*}
    \pause
  \item Mathematical derivation of regression coefficients
    \begin{itemize}
    \item minimum of $Q(\beta_0, \beta_1)$ satisfies $\partial Q
      / \partial \beta_0 = \partial Q / \partial \beta_1 = 0$
    \item leads to normal equations (system of 2 linear equations)
      \begin{footnotesize}
      \begin{align*}
        -2 \sum_{i=1}^n \bigl[y_i - (\beta_0 + \beta_1 x_i)\bigr] 
        &= 0 \quad \so
        &
        \beta_0 \primary{n} 
        + \beta_1 \primary{\sum_{i=1}^n x_i}
        &= \secondary{\sum_{i=1}^n y_i}
        \\
        -2 \sum_{i=1}^n x_i \bigl[y_i - (\beta_0 + \beta_1 x_i)\bigr] 
        &= 0 \quad \so
        &
        \beta_0 \primary{\sum_{i=1}^n x_i}
        + \beta_1 \primary{\sum_{i=1}^n x_i^2}
        &= \secondary{\sum_{i=1}^n x_i y_i}
      \end{align*}
      \end{footnotesize}
    \item regression coefficients = unique solution $\hat{\beta}_0, \hat{\beta}_1$
    \item[]
    \end{itemize}
  \end{itemize}
\end{frame}

\begin{frame}
  \frametitle{The Pearson correlation coefficient}
  %\framesubtitle{}

  \begin{itemize}
  \item Measuring the ``goodness of fit'' of the linear prediction
    \begin{itemize}
    \item variation among observed values of $Y$ = sum of squares $S_y^2$
    \item closely related to (sample estimate for) variance of $Y$
      \[
      S_y^2 = \sum_{i=1}^n (y_i - \bar{y}_n)^2
      \]
    \item residual variation wrt.\ linear prediction: $S^2_{\text{resid}} = Q$
    \item[]
    \end{itemize}
    \pause
  \item Pearson correlation = amount of variation ``explained'' by $X$
    \[
    R^2 = 1 - \frac{S^2_{\text{resid}}}{S_y^2}
    = 1 - \frac{
      \sum_{i=1}^n (y_i - \beta_0 - \beta_1 x_i)^2
    }{
      \sum_{i=1}^n (y_i - \bar{y}_n)^2
    }
    \]
    \begin{itemize}
    \item[\hand] correlation \vs slope of regression line
      \[
      R^2 = \hat{\beta}_1(y \sim x) \cdot \hat{\beta}_1(x \sim y)
      \]
    \end{itemize}
  \end{itemize}
\end{frame}

\subsection{General linear regression}

\begin{frame}
  \frametitle{Multiple linear regression}
  %\framesubtitle{}

  \begin{itemize}
  \item Linear regression with multiple predictor variables
    \[
    Y \approx \beta_0 + \beta_1X_1 + \dots + \beta_k X_k
    \]
    minimises
    \[
    Q = \sum_{i=1}^n \bigl[ y_i - (\beta_0 + \beta_1 x_{i1} + \dots + \beta_k x_{ik}) \bigr]^2
    \]
    for data points $\bigl( x_{11},\ldots,x_{1k},y_1 \bigr),\; \ldots,\; \bigl(
    x_{n1},\ldots,x_{nk}, y_n \bigr)$
  \item[]
  \item Multiple linear regression fits $n$-dimensional hyperplane instead of
    regression line
  \end{itemize}
\end{frame}

\begin{frame}
  \frametitle{Multiple linear regression: The design matrix}
  %\framesubtitle{}

  \begin{itemize}
  \item Matrix notation of linear regression problem
    \[
    \mathbf{y} \approx \mathbf{Z} \mathbf{\beta}
    \]
  \item ``Design matrix'' $\mathbf{Z}$ of the regression data
    \begin{align*}
      \mathbf{Z} &=
      \begin{bmatrix}
        1 & x_{11} & x_{12} & \cdots & x_{1k} \\
        1 & x_{21} & x_{22} & \cdots & x_{2k} \\
        \vdots & \vdots & \vdots &  & \vdots \\
        1 & x_{n1} & x_{n2} & \cdots & x_{nk} \\
      \end{bmatrix}
      \\
      \mathbf{y} &=
      \begin{bmatrix}
        y_1 & y_2 & \ldots & y_n
      \end{bmatrix}'
      \\
      \mathbf{\beta} &=
      \begin{bmatrix}
        \beta_0 & \beta_1 & \beta_2 & \ldots & \beta_k
      \end{bmatrix}'
    \end{align*}
    \ungap[1.5]
    \begin{itemize}
    \item[\hand] $\mathbf{A}'$ denotes transpose of a matrix; $\mathbf{y},
      \mathbf{\beta}$ are column vectors
    \end{itemize}
  \end{itemize}
\end{frame}

\begin{frame}
  \frametitle{General linear regression}
  %\framesubtitle{}

  \begin{itemize}
  \item Matrix notation of linear regression problem
    \[
    \mathbf{y} \approx \mathbf{Z} \mathbf{\beta}
    \]
  \item Residual error
    \[
    Q = (\mathbf{y}-\mathbf{Z}\mathbf{\beta})'
    (\mathbf{y}-\mathbf{Z}\mathbf{\beta})
    \]
    \pause\ungap[1]
  \item System of normal equations satisfying $\nabla_{\beta}\, Q = 0$:
    \[
    \mathbf{Z}' \mathbf{Z} \mathbf{\beta} = \mathbf{Z}' \mathbf{y}
    \]
    \pause\ungap[1]
  \item Leads to regression coefficients
    \[
    \mathbf{\hat{\beta}} = (\mathbf{Z}' \mathbf{Z})^{-1} \mathbf{Z'} \mathbf{y}
    \]
  \end{itemize}
\end{frame}

\begin{frame}
  \frametitle{General linear regression}
  %\framesubtitle{}

  \begin{itemize}
  \item Predictor variables can also be functions of the observed variables
    \so regression only has to be linear in coefficients $\mathbf{\beta}$
  \item[]
  \item E.g.\ polynomial regression with design matrix
    \[
    \mathbf{Z} =
    \begin{bmatrix}
      1 & x_1 & x_1^2 & \cdots & x_1^k \\
      1 & x_2 & x_2^2 & \cdots & x_2^k \\
      \vdots & \vdots & \vdots &  & \vdots \\
      1 & x_n & x_n^2 & \cdots & x_n^k \\
    \end{bmatrix}
    \]
    corresponding to regression model
    \[
    Y \approx \beta_0 + \beta_1 X + \beta_2 X^2 + \dots + \beta_k X^k
    \]
  \end{itemize}
\end{frame}

%%%%%%%%%%%%%%%%%%%%%%%%%%%%%%%%%%%%%%%%%%%%%%%%%%%%%%%%%%%%%%%%%%%%%%
\section{Linear statistical models}

\subsection{A statistical model of linear regression}

\begin{frame}
  \frametitle{Linear statistical models}
  %\framesubtitle{}
  
  \begin{itemize}
  \item Linear statistical model ($\epsilon$ = random error)
    \begin{align*}
      Y &= \beta_0 + \beta_1 x_1 + \cdots + \beta_k x_k + \epsilon \\
      \epsilon &\sim N(0, \sigma^2)
    \end{align*}
    \ungap[1.5]
    \begin{itemize}
    \item[\hand] $x_1, \ldots, x_k$ are not treated as random variables!
    \item $\sim$ = ``is distributed as''; $N(\mu, \sigma^2)$ = normal distribution
    \item[]\pause
    \end{itemize}
  \item Mathematical notation:
    \[
    Y \,|\, x_1, \ldots, x_k 
    \sim N\bigl(\beta_0 + \beta_1 x_1 + \cdots + \beta_k x_k, \sigma^2\bigr)
    \]
    \ungap\pause
  \item Assumptions
    \begin{itemize}
    \item error terms $\epsilon_i$ are i.i.d.\ (independent, same distribution)
    \item error terms follow normal (Gaussian) distributions
    \item equal (but unknown) variance $\sigma^2$ = homoscedasticity
    \end{itemize}
  \end{itemize}
\end{frame}

\begin{frame}
  \frametitle{Linear statistical models}
  %\framesubtitle{}

  \begin{itemize}
  \item Probability density function for simple linear model
    \[
    \pC{\mathbf{y}}{\mathbf{x}}
    = \frac{1}{(2\pi \sigma^2)^{n/2}} \cdot
    \exp \left[
      -\frac{1}{2\sigma^2} \sum_{i=1}^n (y_i - \beta_0 - \beta_1 x_i)^2
    \right]
    \]
    \ungap[1]
    \begin{itemize}
    \item $\mathbf{y}=(y_1,\ldots,y_n)$ = observed values of $Y$ (sample size $n$)
    \item $\mathbf{x}=(x_1,\ldots,x_n)$ = observed values of $X$
    \item[]
    \end{itemize}
    \pause
  \item Log-likelihood has a familiar form:
    \[
    \log \pC{\mathbf{y}}{\mathbf{x}}
    = C - \frac{1}{2\sigma^2} \sum_{i=1}^n (y_i - \beta_0 - \beta_1 x_i)^2
    \propto Q
    \]
  \item[\So] MLE parameter estimates $\hat{\beta}_0, \hat{\beta}_1$ from linear regression
  \end{itemize}
\end{frame}

\subsection{Statistical inference}

\begin{frame}
  \frametitle{Statistical inference for linear models}
  %\framesubtitle{}

  \begin{itemize}
  \item Model comparison with ANOVA techniques
    \begin{itemize}
    \item Is variance reduced significantly by taking a specific
      explanatory factor into account?
    \item intuitive: proportion of variance explained (like $R^2$)
    \item mathematical: $F$ statistic \so $p$-value
    \item[]
    \end{itemize}
    \pause
  \item Parameter estimates $\hat{\beta}_0, \hat{\beta}_1, \ldots$ are random
    variables
    \begin{itemize}
    \item $t$-tests ($H_0: \beta_j = 0$) and confidence intervals for $\beta_j$
    \item confidence intervals for new predictions
    \item[]
    \end{itemize}
    \pause
  \item Categorical factors: dummy-coding with binary variables
    \begin{itemize}
    \item e.g.\ factor $x$ with levels \emph{low}, \emph{med}, \emph{high} is
      represented by three binary dummy variables $x_{\text{low}}, x_{\text{med}}, x_{\text{high}}$
    \item one parameter for each factor level:
      $\beta_{\text{low}}, \beta_{\text{med}}, \beta_{\text{high}}$
    \item NB: $\beta_{\text{low}}$ is ``absorbed'' into intercept $\beta_0$\\
      model parameters are usually $\beta_{\text{med}} - \beta_{\text{low}}$
      and $\beta_{\text{high}} - \beta_{\text{low}}$
    \item[\hand] mathematical basis for standard ANOVA
    \end{itemize}
  \end{itemize}
\end{frame}

\begin{frame}
  \frametitle{Interaction terms}
  %\framesubtitle{}

  \begin{itemize}
  \item Standard linear models assume independent, additive contribution from
    each predictor variable $x_j$ ($j = 1, \ldots, k$)
  \item Joint effects of variables can be modelled by adding interaction terms
    to the design matrix (+ parameters)
  \item Interaction of numerical variables (interval scale)
    \begin{itemize}
    \item interaction term for variables $x_i$ and $x_j$ = product $x_i\cdot x_j$
    \item e.g.\ in multivariate polynomial regression:\\
      $Y = p(x_1,\ldots, x_k) + \epsilon$ with polynomial $p$ over $k$
      variables
    \end{itemize}
  \item Interaction of categorical factor variables (nominal scale)
    \begin{itemize}
    \item interaction of $x_i$ and $x_j$ coded by one dummy variable for each
      combination of a level of $x_i$ with a level of $x_j$
    \item alternative codings e.g.\ to have separate parameters for
      independent additive effects of $x_i$ and $x_j$
    \end{itemize}
  \item Interaction of categorical factor with numerical variable
  \end{itemize}
\end{frame}

%%%%%%%%%%%%%%%%%%%%%%%%%%%%%%%%%%%%%%%%%%%%%%%%%%%%%%%%%%%%%%%%%%%%%% 
\section{Generalised linear models}

\begin{frame}
  \frametitle{Generalised linear models}
  %\framesubtitle{}

  \begin{itemize}
  \item Linear models are flexible analysis tool, but they \ldots
    \begin{itemize}
    \item only work for a numerical response variable (interval scale)
    \item assume independent (i.i.d.) Gaussian error terms
    \item assume equal variance of errors (homoscedasticity)
    \item cannot limit the range of predicted values
    \end{itemize}
  \item Linguistic frequency data problematic in all four respects
    \begin{itemize}
    \item[\hand] each data point $y_i$ = frequency $f_i$ in one text sample
    \item $f_i$ are discrete variables with binomial distribution (or more
      complex distribution if there are non-randomness effects)
    \item[\hand] linear model uses relative frequencies $p_i = f_i / n_i$
    \item Gaussian approximation not valid for small text size $n_i$
    \item sampling variance depends on text size $n_i$ and ``success
      probability'' $\pi_i$ (= relative frequency predicted by model)
    \item model predictions must be restricted to range $0\leq p_i\leq 1$
    \end{itemize}
  \item[\So] General\emph{ised} linear models (GLM) 
  \end{itemize}
\end{frame}

\begin{frame}
  \frametitle{Generalised linear model for corpus frequency data}
  %\framesubtitle{}

  \begin{itemize}
  \item Sampling family (binomial)
    \[
    f_i \sim B(n_i, \pi_i)
    \]
    \pause\ungap[1]
  \item Link function (success probability $\pi$ $\leftrightarrow$ odds $\theta$)
    \[
    \pi_i = \frac{1}{1 + e^{-\theta_i}}
    \]
    \pause\ungap[1]
  \item Linear predictor
    \[
    \theta_i = \beta_0 + \beta_1 x_{i1} + \cdots + \beta_k x_{ik}
    \]
    \pause\ungap[1]
  \item[\So] Estimation and ANOVA based on likelihood ratios
    \begin{itemize}
    \item[\hand] iterative methods needed for parameter estimation
    \end{itemize}
  \end{itemize}
\end{frame}

%%%%%%%%%%%%%%%%%%%%%%%%%%%%%%%%%%%%%%%%%%%%%%%%%%%%%%%%%%%%%%%%%%%%%%
% \section{}

% \begin{frame}
%   \frametitle{}
%   %\framesubtitle{}

% \end{frame}

%%%%%%%%%%%%%%%%%%%%%%%%%%%%%%%%%%%%%%%%%%%%%%%%%%%%%%%%%%%%%%%%%%%%%%
%% References (if any)

%% \frame[allowframebreaks]{
%%   \frametitle{References}
%%   \bibliographystyle{natbib-stefan}
%%   \begin{scriptsize}
%%     \bibliography{stefan-publications,stefan-literature}
%%   \end{scriptsize}
%% }

\end{document}
