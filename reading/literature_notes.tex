\documentclass[a4paper]{article}

\usepackage{vmargin}
\setpapersize[portrait]{A4}
\setmarginsrb{30mm}{10mm}{30mm}{20mm}% left, top, right, bottom
{12pt}{15mm}% head heigth / separation
{0pt}{15mm}% bottom height / separation
%% \setmargnohfrb{30mm}{20mm}{20mm}{20mm}

\setlength{\parindent}{0mm}
\setlength{\parskip}{\medskipamount}

\usepackage[english]{babel}
\usepackage[utf8]{inputenc}

\usepackage[T1]{fontenc}
%% \usepackage{textcomp}  % this can break some outline symbols in CM fonts, use only if absolutely necessary

%% \usepackage{lmodern}   % type1 computer modern fonts in T1 encoding
% \usepackage{stefan-fonts}  % commercial Charter fonts with full math support
\usepackage{chmath}

%% \usepackage{mathptmx}  % use Adobe Times as standard font with simulated math support
%% \usepackage[sc,osf]{mathpazo}  % use Adobe Palatino as standard font with simulated math support

%% \usepackage{pifont}
%% \usepackage{eucal}

\usepackage{amsmath,amssymb,amsthm}
\usepackage{graphicx,rotating}
\usepackage{array,hhline,booktabs}
\usepackage{xspace}
\usepackage{url}
%% \usepackage{ifthen,calc,hyphenat}

\usepackage{xcolor,tikz}
\usepackage[textwidth=25mm,textsize=footnotesize,colorinlistoftodos,backgroundcolor=orange!80]{todonotes} % [disable] to hide all TODOs

\usepackage{natbib}
\bibpunct{(}{)}{;}{a}{}{,}

%%
%%  INCLUDE: math.tex
%%  
%%  basic mathematical symbols and constructs (not specific to cooccurrences)
%%


%% \setN, \setN[0], \setZ, \setQ, \setR, \setC
%% abbreviations for common number spaces
\newcommand{\setN}[1][]{\mathbb{N}_{#1}} % allows \setN and \setN[0]
\newcommand{\setZ}{\mathbb{Z}}
\newcommand{\setQ}{\mathbb{Q}}
\newcommand{\setR}{\mathbb{R}}
\newcommand{\setC}{\mathbb{C}}

%% \set{el_1, el_2, ...};  \setdef{el}{condition};  \bigset{..}, \bigsetdef{..}{..}
%% extensional and intensional definition of sets, with "big" versions (like \bigl etc.)
\newcommand{\set}[1]{\left\{#1\right\}}
\newcommand{\setdef}[2]{\set{#1\,\left|\,#2\right.}}
\newcommand{\bigset}[1]{\bigl\{#1\bigr\}}
\newcommand{\bigsetdef}[2]{\bigset{#1\bigm|#2}}
\newcommand{\Bigset}[1]{\Bigl\{#1\Bigr\}}
\newcommand{\Bigsetdef}[2]{\Bigset{#1\Bigm|#2}}
\newcommand{\biggset}[1]{\biggl\{#1\biggr\}}
\newcommand{\biggsetdef}[2]{\biggset{#1\biggm|#2}}

%% \compl{X} = complement of set X
\newcommand{\compl}[1]{\mathcal{C} #1}

%% \eps == \epsilon, \si == \sigma, \sisi == \sigma^2, \ka == \kappa
\newcommand{\eps}{\epsilon}
\newcommand{\si}{\sigma}
\newcommand{\sisi}{\sigma^2}
\newcommand{\ka}{\kappa}

%% \abs{expr}, \bigabs{expr}, \norm{expr}, \bignorm{expr}
%% absolute value and norm of expression, with "big" versions
\newcommand{\abs}[1]{\left\lvert#1\right\rvert}
\newcommand{\bigabs}[1]{\bigl\lvert#1\bigr\rvert}
\newcommand{\norm}[1]{\left\lVert#1\right\rVert}
\newcommand{\bignorm}[1]{\bigl\lVert#1\bigr\rVert}

%% \constpi == constant PI (in bold font)
\newcommand{\constpi}{\boldsymbol{\pi}}

%% \dx == "dx";  \dx[z] == "dz";  \dpi == \dx[\pi];  
%% \dG == \dx[G], \dt == \dx[t]
\newcommand{\dx}[1][x]{\,d#1}
\newcommand{\dpi}{\dx[\pi]}
\newcommand{\dG}{\dx[G]}
\newcommand{\dt}{\dx[t]}

%% \Int{\frac{1}{2} x^2}_a^b
%% anti-derivative evaluated to compute definite integral
\newcommand{\Int}[1]{\left[#1\right]}

%% \limdownto{x}{0}
%% limit from above for x -> 0
\newcommand{\limdownto}[2]{\lim_{#1\,\downarrow\,#2}}

%% \iffdef == ":<=>";  \iffdefR == "<=>:"
\newcommand{\iffdef}{\;:\!\iff}
\newcommand{\iffdefR}{\iff\!:\;}

%% \logten(x) 
%% base 10 logarithm, which is always used in the UCS system
\newcommand{\logten}{\log_{10}}

%% \e+3, \e-6, \e-{12}, 5.5\x\e-3
%% engineering-style notation (orders of magnitude) for floating-point numbers
\newcommand{\e}[2]{10^{\ifthenelse{\equal{#1}{+}}{}{#1}#2}}
\newcommand{\x}{\cdot}

%% \Landau{ n^2 }, \bigLandau{ N^2 }
%% Landau symbol ("big oh notation")
\newcommand{\Landau}[1]{\mathcal{O}\left({#1}\right)}
\newcommand{\bigLandau}[1]{\mathcal{O}\bigl({#1}\bigr)}


%%% Local Variables: 
%%% mode: latex
%%% TeX-master: t
%%% End: 

%%
%%  INCLUDE: stat.tex
%%  
%%  symbols and notation for probability theory and statistics
%%


%% \p{X=k};  \pC{X=k}{Y=l};  \pscale{\frac{Z}{S^2}}
%% probability P(X=k), conditional probability P(X=k|Y=l), and variants with scaled parentheses
\newcommand{\p}[1]{\mathop{\mathrm{Pr}}\bigl(#1\bigr)}
\newcommand{\pscale}[1]{\mathop{\mathrm{Pr}}\left(#1\right)}
\newcommand{\pC}[2]{\p{#1\bigm|#2}} 
\newcommand{\pCscale}[2]{\pscale{#1\,\left|\,#2\right.}} 

%% \Exp{X};  \Var{X};  \Exp[0]{X};  \Var[0]{X};  \Expscale{X};  \Varscale{X}
%% expectation E[X] and variance V[X], expectation and variance under null hypothesis, 
%% and variants with scaled brackets
\newcommand{\Exp}[2][]{E_{#1}\bigl[#2\bigr]}
\newcommand{\Var}[2][]{\mathop{\mathrm{Var}}_{#1}\bigl[#2\bigr]}
\newcommand{\Expscale}[2][]{E_{#1}\left[#2\right]}
\newcommand{\Varscale}[2][]{\mathop{\mathrm{Var}}_{#1}\left[#2\right]}

%% \I{f_i = M};  \bigI{\sum_{i=1}^S f_i = N}
%% indicator variable (as in Baayen 2001), and variant with explicitly scaled brackets
\newcommand{\I}[1]{I_{\left[#1\right]}}
\newcommand{\bigI}[1]{I_{\bigl[#1\bigr]}}

%% \Hind;  \Hhom;  \Hnull{\kappa = x}
%% null hypothesis of independence and homogeneity; general null hypothesis identified by condition
\newcommand{\Hind}{H_0}
\newcommand{\Hhom}{H_{0,\, hom}}
\newcommand{\Hnull}[1]{H_{#1}}

%% \confint{\kappa};  \confint[0.99]{\kappa}
%% confidence interval for specified population characteristic (conf. level defaults to \alpha)
\newcommand{\confint}[2][\alpha]{I_{#2,\,#1}}

%% \df = 1
%% degrees of freedom
\newcommand{\df}{\mathit{df}}


%%% Local Variables: 
%%% mode: latex
%%% TeX-master: t
%%% End: 


\title{SIGIL: Statistical Inference for Corpus Linguistics\\--- literature review \& notes ---}
\author{Stefan Evert}
\date{Typeset on \today}

\begin{document}
\maketitle

\listoftodos
\tableofcontents


%%%%%%%%%%%%%%%%%%%%%%%%%%%%%%%%%%%%%%%%%%%%%%%%%%%%%%%%%%%%%%%%%%%%%%%%
%%%%%%%%%%%%%%%%%%%%%%%%%%%%%%%%%%%%%%%%%%%%%%%%%%%%%%%%%%%%%%%%%%%%%%%%
\section{Inbox \& tasks}
\label{sec:inbox}

%%%%%%%%%%%%%%%%%%%%%%%%%%%%%%%%%%%%%%%%%%%%%%%%%%%%%%%%%%%%%%%%%%%%%%%%
\subsection{Tasks for \texttt{corpora} package}
\label{sec:tasks-corpora}

\begin{itemize}
\item Change implementation of p-values and confidence intervals to use ``central'' method \citep[$p_c$, cf.][]{Fay:10a}, which is efficient and consistent.  Explain in help pages why this is a sensible choice for corpus linguistics (see Sec.~\ref{sec:Fay2010}).
\end{itemize}

%%%%%%%%%%%%%%%%%%%%%%%%%%%%%%%%%%%%%%%%%%%%%%%%%%%%%%%%%%%%%%%%%%%%%%%%
\subsection{Tasks for SIGIL slides}
\label{sec:tasks-sigil}

\begin{itemize}
\item Mention different types of two-sided p-values and confidence intervals, which can lead to inconsistencies, with example from \citet{Fay:10a}.  Recommendation for corpus linguistics: conservative central p-values $p_c$ and matching confidence intervals, implemented in \texttt{corpora} package.  Note that most standard implementations (e.g.\ \texttt{fisher.test} and \texttt{binom.test} in R) calculate minlike p-values $p_m$ that will often differ from $p_c$.
\end{itemize}

%%%%%%%%%%%%%%%%%%%%%%%%%%%%%%%%%%%%%%%%%%%%%%%%%%%%%%%%%%%%%%%%%%%%%%%%
\subsection{Inbox}
\label{sec:inbox}

%%%%%%%%%%%%%%%%%%%%%%%%%%%%%%%%%%%%%%%%%%%%%%%%%%%%%%%%%%%%%%%%%%%%%%%%
\subsubsection{\citet{Fay:10,Fay:10a}: Exact confidence intervals for two-sided tests}
\label{sec:Fay2010}

\begin{itemize}
\item Notation: test statistic $T$ for parameter $\theta$ with observed value $t$; $f_{\theta}(t) = \pC{T=t}{\theta}$ is the likelihood; $F_{\theta}(t) = \pC{T\leq t}{\theta}$ and $\bar{F}_{\theta}(t) = \pC{T\geq t}{\theta}$ are the lower/upper tail probabilities
\item My notaton: $G_{\theta}(t) = \min\set{F_{\theta}(t), \bar{F}_{\theta}(t)}$ for the ``appropriate'' tail probability
\item Exact two-sided p-values for discrete data can be defined in different ways. \citet{Fay:10a} lists three important and widely-used methods:
  \begin{enumerate}
  \item \textbf{central} p-value: $p_c = 2\cdot G_{\theta}(t)$, clamped to $[0,1]$ (aka.\ TST = twice the smaller tail); usually easy to compute and well-behaved (monotonic in $\theta$, cf.\ Fig.~1 on p.~56); exact confidence intervals are often based on $p_c$
  \item \textbf{minlike} p-value: $p_m = \sum_{f(s)\leq f(t)} f_{\theta}(s)$; implementations of exact tests often report $p_m$ as two-sided p-value; not always well-behaved and confidence sets may have ``holes''
  \item \textbf{blaker} p-value: $p_b$ is the appropriate tail probability $G_{\theta}(t)$ plus the largest opposite tail that does not exceed $G_{\theta}(t)$ (aka.\ CT = combined tails method); \citet{Blaker:00}\todo{get copy of \citet{Blaker:00} and update details in bibtex database} presents a comprehensive analysis of this method and derives improved confidence intervals
  \end{enumerate}
\item \citet{Fay:10a} gives an excellent concise overview of two-sided p-values and confidence intervals for $2\times 2$ tables using the three methods above, with examples of inconsistencies arising if different methods are mixed (typically central confidence interval with minlike p-value)
\item All three methods and Matching confidence intervals developed by \citet{Fay:10}\todo{read \citet{Fay:10} and add notes} are implemented in the R packages \textbf{exact2x2} and \textbf{exactci}; in some cases, inconsistencies cannot be entirely avoided because ``true'' confidence sets aren't connected intervals (Fig.~3, p.~57)
\item Conclusion: central p-value $p_c$ is easy to compute, well-behaved (Fig.~1) and consistent confidence intervals can be determined efficiently; in most cases (though not always) it is more conservative than the other methods (i.e.\ $p_c > p_m, p_b$).  Therefore it makes sense to \textbf{use central p-values} and the matching confidence intervals in \textbf{corpus linguistics}
\end{itemize}

%%%%%%%%%%%%%%%%%%%%%%%%%%%%%%%%%%%%%%%%%%%%%%%%%%%%%%%%%%%%%%%%%%%%%%%%
% \subsection{}



%%%%%%%%%%%%%%%%%%%%%%%%%%%%%%%%%%%%%%%%%%%%%%%%%%%%%%%%%%%%%%%%%%%%%%%%  
%%%%%%%%%%%%%%%%%%%%%%%%%%%%%%%%%%%%%%%%%%%%%%%%%%%%%%%%%%%%%%%%%%%%%%%%
% \section{}

%%%%%%%%%%%%%%%%%%%%%%%%%%%%%%%%%%%%%%%%%%%%%%%%%%%%%%%%%%%%%%%%%%%%%%%%
% \subsection{}

%%%%%%%%%%%%%%%%%%%%%%%%%%%%%%%%%%%%%%%%%%%%%%%%%%%%%%%%%%%%%%%%%%%%%%%%
% \subsubsection{}


%% \renewcommand{\bibsection}{}    % avoid (or change) section heading 
%% \bibliographystyle{apalike}
\bibliographystyle{natbib-stefan}
\bibliography{stefan-literature,stefan-publications}  

\end{document}
